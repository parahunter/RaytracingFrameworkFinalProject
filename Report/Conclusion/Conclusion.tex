We have demonstrated two different ways of performing the sampling for ambient occlusion that produced two different visual results. From the litterature it seems that both methods are acceptable so it is a matter of what visual style that fits the desired result the best. That the two different methods produce two overall different levels of brightness. A likely candidate is the lack of a modification with the probability density function in the rejection sampling method but it does not seem to be something that other papers take into consideration \cite{Gems17}. The lookup into the light probe image and conversion from the RGBE format to floating point RGB seems to work in a satisfactory way. Here a better mapping to values the screen can show could also improve the result.
\\ \\
The two most obvious improvements to our method would be to change the weight of each occluded ray based on how far away the hit on the occluder are. Another improvement would be to implemnt a blurred lookup into the envirornment map.