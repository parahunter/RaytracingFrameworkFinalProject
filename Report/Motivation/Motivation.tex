Even though global illumination techniques have come a long way they still have many limitations. One of them is the often long render times they require and the difficulties of implementing and working with them efficiently. Ambient occlusion is a rather simple technique that allows indirect lighting from an envirornment to be simulated cheaply. The technique was originally developed by Hayden Landis and colleagues at Industrial light and Magic to accomodate the needs they had for production ready global illumination \cite{Landis2002}. Ambient occlusion together with envirornment mapping is therefore often used in the industry since they besides being faster also allows for easy compositing of synthetic 3D images into live video footage\cite{Landis2002}. Ambient occlusion is therefore an often used lighting technique in the movie industry.
\\ \\
We will in this report present an implementation of Ambient Occlusion into the Raytracing framework used for the course " - Rendering"