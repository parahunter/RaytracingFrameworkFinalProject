Even though global illumination techniques have come a long way they still have many limitations. One of them is the often long render times they require. Ambient occlusion is a rather simple technique that allows indirect lighting from an envirornment to be simulated cheaply. The technique was originally developed by Hayden Landis and colleagues at Industrial light and Magic to accomodate the needs they had for production ready global illumination \cite{Landis2002}. Ambient occlusion together with envirornment mapping has since then often been used in the industry. The reason for this is that ambient occlusion besides being faster also allows for easy compositing of synthetic 3D images into live video footage\cite{Landis2002}. The technique is also often sought implemented in a realtime context since its cheap evaluation lends itself well to realtime graphics\cite{Kontkanen:2005,Umenhoffer:2009, Shanmugam:2007}.
\\ \\
We will in this report present an implementation of Ambient Occlusion into the Raytracing framework used for the course "02562 - Rendering Introduction" at Technical University of Denmark.